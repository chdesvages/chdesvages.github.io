\mytitle{Anabelian stratified spaces}

Let $G$ be a pyknotic group, and
let $X \to BG$ be a pyknotic stratified space over $BG$.
In other words, the fiber $\overline{X}$ over ``the'' basepoint of $BG$ is
a pyknotic stratified space with a homotopy action of $G$.

From this data, we can construct a genuine $G$-space.
For the orbit category $O_G$ of $G$, I guess I'll take%
\sidenote{One question is whether this is really the correct orbit category.
	I think it depends on the kinds of examples one wants.
	In my case, I think this will do.}
the category of connected pyknotic spaces over $BG$ whose fibers are finite and discrete.
So given my $X$, the $G$-space I want to consider carries
$T \to BG$ to the space $N\Fun_{BG}(T, X)$.
If we think of $T=BH$, we're saying the $H$-fixed points are
$N(\overline{X}^{hH})$.

Since the nerve%
\sidenote{AKA the ``invert-everything'' functor}
doesn't preserve homotopy fixed points,
there is no guarantee that this $G$-space is cofree --
i.e., that the $H$-fixed points are homotopy fixed points.%
\sidenote{What $G$-spaces can one get by this sort of move?
	My temptation is to say that you can model \emph{any} $G$-space in this way.
	But I don't really have any idea.
	It feels kind of Thomason-ish.}

So here's the definition.
I'll call $X$ \defn{anabelian} if and only if this $G$-space is cofree.

Grothendieck's Section Conjecture says that
if $X=\Gal(C)$ for a high-genus complete curve over a number field $K$,
then $X \to BG_K$ is anabelian.

This rhymes with the Sullivan conjecture.
I don't know how to make that sentiment precise.
But if I understood the proof of the Sullivan conjectures,
could I in fact understand this?





