\mytitle{Absolute schemes in characteristic $p$}

I guess the question is
which approach to Drinfel'd's lemma do I want to take seriously?

Sometimes people want to regard it as a statement about the product
$(X/F^{\ZZ})\times(Y/F^{\ZZ})$;
others want to see it as a statement about the product
$X \times_{\FF_1} Y$,
where that's now my notation for the product in sheaves on $\Perf/\TT$.

Maybe what I want to say is that the étale picture of these absolute schemes is not different from the étale picture of them as ordinary schemes,
but there might be some finer topology that ``sees'' the extra loop coming from the Frobenius.

This seems connected to the idea that Ben Zvi suggested:
he pointed out that we should probably expect some dimensional reduction from $4$ to $3$ when considering field theories on $\Spec O_K$.
As BZ suggested, the geometrization of global Langlands is probably an equivalence of two $4$-dimensional field theories, 
and the way that this makes sense on our $3$-dimensional number rings is a dimensional reduction.
That dimensional reduction, in effect, is probably a product with a circle.

This is just a naïve dimensional analysis, but
it might be made sensible.
For instance, we could start by understanding something very elementary --
like $\Spec E$ for $E$ an algebraic closure of a finite field.

It seems as though I'm asserting that there's a more refined topology on $\Spec E$
whose homotopy type is a circle.
What then are the points of this new topology?
Some additional structures that see that $F$ is nontrivial on $E$,
despite the fact that it's trivial on the Zariski or étale topoi.

Such a structure must, in some sense, measure the size of a field.
