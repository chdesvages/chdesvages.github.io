%!TEX root = book.tex 
% chktex-file 1
% chktex-file 3
% chktex-file 8
% chktex-file 12
% chktex-file 18
% chktex-file 24
% chktex-file 35 
% chktex-file 42

%-------------------------------------------------------------------%
%-------------------------------------------------------------------%
%-------------------------------------------------------------------%
\chapter*{Introduction}%
\label{cha:introduction}
%-------------------------------------------------------------------%
%-------------------------------------------------------------------%
%-------------------------------------------------------------------%

When we perform completion constructions,
these involve operations − such as the formation of limits −
that are not compatible with many colimits.
As a result, these constructions produce derived functors.

\begin{eg*}
	Let $ R $ be a commutative ring,
	and let $ I \subset R $ be an ideal.
	Then the $ I $-adic completion $ C_I $ of modules
	has left derived functors $ \LL^n C_I $.
	These fit into short exact sequences
	\[
		0 \to {\lim_{k}}^1 \Tor^R_{n+1}(R/I^k, M) \to
		\LL^n C_I(M) \to \lim_k \Tor^R_n(R/I^k, M)
		\to 0 \period
	\]
\end{eg*}





