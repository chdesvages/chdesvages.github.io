%!TEX root = book.tex 
% chktex-file 1
% chktex-file 3
% chktex-file 8
% chktex-file 12
% chktex-file 18
% chktex-file 24
% chktex-file 35 
% chktex-file 42

%-------------------------------------------------------------------%
%-------------------------------------------------------------------%
\section{Ultrafilters}%
\label{sec:ultrafilters}
%-------------------------------------------------------------------%
%-------------------------------------------------------------------%

%-------------------------------------------------------------------%
\subsection{Codensity monads}%
\label{sub:codensity_monads}
%-------------------------------------------------------------------%

The codensity monad of a functor $ f \colon A \to B $
is the right Kan extension $ \beta(f) $ of $ f $ along itself,
when it exists.
For formal reasons, this is always a monad on $ B $.

The full functoriality
of the construction $ f \mapsto \beta(f) $
is relevant to us.
In effect, we regard functors as the objects of a category,
and the morphisms are lax-commutative squares.


\begin{definition}
	Let $ A $ and $ B $ be categories.
	Then a bifibration 
	\citep[\S\S 2.4.7]{Lurie2009}
	$ X \to B \times A $ is \defn{representable}
	if and only if,
	for every object $ a \in A $,
	the fiber $ X_a $ has a terminal object.
\end{definition}

\begin{nul}
	Let $ A $ and $ B $ be categories.
	A bifibration $ X \to B \times A $ corresponds
	to a functor $ B^{\op} \times A \to \SS_V $,
	or equivalently to a functor $ \Xi \colon A \to \PP(B) $.
	A representable bifibration is one in which
	each presheaf $ \Xi(a) $ is representable.
	In this way,
	the category of representable bifibrations to $ B \times A $
	is equivalent to the category $ \Fun(A, B) $.

	One can be explicit about the correspondence:
	if $ f \colon A \to B $ is a functor,
	then the corresponding representable bifibration is
	\[
		\Fun(\Delta^1, B) \times_B A \to B \times A \comma
	\]
	and every representable fibration is of this form.
\end{nul}

\begin{construction}
	Let $ \Lax\Cat $ be the full subcategory
	of $ \Fun(\Lambda^1_0, \Cat ) $
	spanned by those diagrams $ A \ot X \to B $
	such that $ X \to B \times A $ is a representable bifibration.

	The objects can be identified with functors $ f \colon A \to B $,
	but $ \Lax\Cat $ is not equivalent to 
	the category $ \Fun(\Delta^1, \Cat) $.
	If $ f \colon A \to B $ and $ g \colon C \to D $
	are functors, then
	a morphism $ \sigma \colon f \to g $ of $ \Lax\Cat $
	determines a functor
	\[
		\Fun(\Delta^1, B) \times_B A \to
		\Fun(\Delta^1, D) \times_D C \period
	\]
	If $ a \in A $ and $ b \in B $ are objects,
	then $ \sigma $ determines a map
	\[
		\Map_B(b, f(a)) \to \Map_D(\psi(b), g(\phi(a))) \period
	\]
	When $ b = f(a) $,
	the image of the identity under this map
	is thus a morphism $ \sigma_a \colon \psi(f(a)) \to g(\phi(a)) $.
	Thus the morphism $ \sigma $ amounts to a lax-commutative square:
	\begin{equation}\label{square:laxsquare}
		\begin{tikzcd}
			A \arrow[r, "f" above]
			  \arrow[d, "\phi" left]
			  & B \arrow[d, "\psi" right]
			      \arrow[dl, phantom, "\scriptstyle \sigma"
			      below right, "\Longleftarrow" sloped] \\ 
			C \arrow[r, "g" below] & D
		\end{tikzcd}
	\end{equation}

	We have two functors
	$ s, t \colon \Lax\Cat \to \Cat $
	which carry a diagram $ [ A \ot X \to B ] $
	to $ A $ and $ B $, respectively.
	We have an equivalence
	\[
		\{A\} \times_{\Cat} \Lax\Cat \times_{\Cat} \{B\}
		\simeq \Fun(A,B) \period
	\]

	In fact, the functor
	$ H_B \colon \Cat^{\op} \to \Cat $
	represented by $ B $
	and the functor
	$ H^A \colon \Cat \to \Cat $
	corepresented by $ A $
	correspond under straightening/unstraightening
	to the cartesian fibration
	\[
		\Lax\Cat \times_{\Cat} \{B\} \to \Cat
	\]
	and the cocartesian fibration
	\[
		\{A\} \times_{\Cat} \Lax\Cat \to \Cat \comma
	\]
	respectively.
\end{construction}

\begin{definition}
	We call $ \Lax\Cat $ the \defn{lax arrow category of categories}.
	If $ C $ is a fixed category,
	then we call 
\end{definition}

\begin{construction}
	Let $ C $ be a category.
	We write $ \Endofun(C) $ for
	the monoidal category of endofunctors of $ C $,
	with the monoidal structure given by composition.

	acts on the left on the category $ \Fun(D, C) $.
	Both the monoidal structure and
	the left module structure
	are given by composition.

	We consider the category $ \LL\Mod_{\Endofun(C)}(\Fun(D,C)) $
	whose objects can be regarded as pairs $ (T, f ) $
	consisting of an algebra $ T \in \Alg(\Endofun(C)) $
	and a $ T $-module $ f $ in $ \Fun(D,C) $.
	Thus 

	For any functor $ f \colon D \to C $,
	we consider the monoidal category $ \Endofun(C)[f] $
	constructed in
	\cite[Definition 4.7.1.1]{Lurie2017}.
	The objects of the category $ \Endofun(C)[f] $
	are pairs $ (T, \eta) $ consisting of
	an object $ T \in \Endofun(C) $
	and a natural transformation $ T \circ f \to f $.
	The assignment $ (T, \eta) \mapsto T $ defines a 
	monoidal forgetful functor
	$ \Endofun(C)[f] \to \Endofun(C) $.

	The terminal object
	(if it exists)
	of $ \Endofun(C)[f] $ is automatically
	an algebra object $ B(f) = (\beta(f), \epsilon)$.
	The image of $ B(f) $ under
	the forgetful functor $ \Endofun(C)[f] \to \Endofun(C) $ is
	the algebra object
	\[
		\beta(f) \in \Alg(\Endofun(C)) \semicolon
	\]
	in other words, $ \beta(f) $ is a monad on $ C $.
	
	If $ \Endofun(C)[f] $ has
	a terminal object $ B(f) = (\beta(f), \epsilon)$,
	then the monad $ \beta(f) $ will be called the
	\defn{codensity monad}
	attached to $ f \colon D \to C $.

	If the category $ \Endofun(C)[f] $ has a terminal object,
	then $ \epsilon $ exhibits
	$ \beta(f) $ as the right Kan extension of $ f $
	along itself.
	Conversely, if the right Kan extension
	of $ f $ along itself exists,
	then that Kan extension defines a terminal object of
	the category $ \Endofun(C)[f] $.

	\cite[\S\S 4.7.1]{Lurie2017} identifies three categories:
	\[
		\LL\Mod(\Fun(D,C)) \times_{\Fun(D,C)} \left\{f\right\} \simeq
		\Alg(\Endofun(C)[f]) \simeq
		\Alg(\Endofun(C))_{/\beta(f)} \period
	\]
	More informally, we may say that
	a morphism of monads $ T \to \beta(f) $
	is the same thing as a $ T $-module structure on $ f $.
\end{construction}

%-------------------------------------------------------------------%
\subsection{Ultrafilters on sets}%
\label{sub:ultrafilters_on_sets}
%-------------------------------------------------------------------%

\begin{notation}
	Let us write $ \Fin \subset \Set $ for
	the full subcategory of finite sets,
	and let us write $ i $ for
	the inclusion $ \Fin \inclusion \Set $.
\end{notation}

\begin{definition}
	Let $ S $ be a set.
	An \defn{ultrafilter} on $ S $ is
	a natural transformation
	\[
		\yo^S \circ i \to i \period
	\]
\end{definition}

\begin{notation}
	Let $ S $ be a set, and
	let $ \mu $ be an ultrafilter on $ S $.
	We find it expressive to write
	\[
		\int_S (\cdot) \ d\mu \colon \yo^S \circ i \to i
	\]
	for the natural transformation.
	If $ I $ is a finite set,
	and if $ f \colon S \to I $ is a map,
	then this natural transformation
	permits us to specify an element
	\[
		\int_S f \ d\mu \in I \period
	\]
	The naturality is the condition that
	if $ \phi \colon I \to J $ is a map of finite sets,
	then
	\[
		\phi \left( \int_S f \ d\mu \right) = 
		\int_S \phi \circ f \ d\mu \period
	\]
\end{notation}

\begin{eg}
	Let $ S $ be a set, and let $ s \in S $ be an element.
	The \defn{principal ultrafilter} $ \delta_s $ is then defined so that
	\[
		\int_S f \ d \delta_s = f(s) \period
	\]
	By the Yoneda lemma,
	every ultrafilter on a finite set is principal,
	but as we shall see,
	infinite sets have ultrafilters that are not principal.
\end{eg}

\begin{notation}
	If $ S $ is a set, then
	let us write $ \beta(S) $ for the set
	of ultrafilters on $ S $:
	\[
		\beta(S) \coloneq \Map(\yo^S \circ i, i) \period
	\]
\end{notation}
	Write $ \beta (S) $ for the set of ultrafilters on $ S $.
	For any set $ S $, the set $ \beta(S) $ is the set
	\[
		\beta(S) = \lim_{I \in \Fin_{S/}} I \period
	\]
	The functor
	\[
		\beta \colon \Set \to \Set
	\]
	is thus the right Kan extension of the inclusion $ \Fin \inclusion \Set $ along itself.

To prove the existence of these, let us look at a more traditional way of defining an ultrafilter on a set.

\begin{definition}
	Let $ S $ be a set, $ T \subseteq S$, and $ \mu $ an ultrafilter on $ S $.
	There is a unique \defn{characteristic map} $ \chi_T \colon S \to \{ 0,1 \}$ such that $ \chi_T(s) = 1 $ if and only if $ s \in T $.
	Let us write
	\[
		\mu(T) \coloneq \int_S \chi_T \ d \mu \period
	\]
	
	We say that \defn{$ T $ is $ \mu $-thick} if and only if $\mu(T) = 1$.
	Otherwise (that is, if $ \mu(T) = 0 $), then we say that $ T $ is \defn{$ \mu $-thin}.

	For any $ s \in S$, the principal ultrafilter $ \delta_s $ is the unique ultrafilter relative to which $ \{ s \} $ is thick.
\end{definition}

\begin{scholium}
	If $ S $ is a set and $ \mu $ is an ultrafilter on $ S $, then we can observe the following facts about the collection of thick and thin subsets (relative to $ \mu $):
	\begin{enumerate}
		\item The empty set is thin.
		\item Complements of thick sets are thin.
		\item Every subset is either thick or thin.
		\item Subsets of thin sets are thin.
		\item The intersection of two thick sets is thick.
	\end{enumerate}
	In other words, if $ S $ is a set, then an ultrafilter on $ S $ is tantamount to a Boolean algebra homomorphism $ \PP(S) \to \{0,1\} $.

	It is possible to define ultrafilters on more general posets, and if $ P $ is a Boolean algebra, then an ultrafilter is precisely a Boolean algebra homomorphism $ P \to \{0, 1\} $.
\end{scholium}

\begin{scholium}
	Ultrafilters are functorial in maps of sets.
	Let $ \phi \colon S \to T $ be a map, and let $ \mu $ be an ultrafilter on $ S $.
	The ultrafilter $ \phi_{\ast}\mu $ on $ T $ given by
	\[
		\int_T f \ d (\phi_{\ast}\mu) = \int_S (f \circ \phi) \ d \mu \period
	\]
	For any $ U \subseteq T$, one has in particular
	\[
		(\phi_{\ast} \mu)(U) = \mu (\phi^{-1}(U)) \period
	\]
	Thus $ U $ is $ \phi_{\ast} \mu $-thick if and only if $ \phi^{-1} U $ is $ \mu $-thick.
\end{scholium}

\begin{definition}
	A \defn{system of thick subsets} of $ S $ is a collection $ F \subseteq \PP(S) $ such that for any finite set $ I $ and any partition
	\[
		S = \coprod_{ i \in I } S_i \comma
	\]
	there is a unique $ i \in I $ such that $ S_i \in F $.
\end{definition}

\begin{construction}
	We have seen that an ultrafilter $ \mu $ specifies the system $ F_{\mu} $ of $ \mu $-thick subsets.
	In the other direction, attached to any system $F$ of thick subsets is an ultrafilter $\mu_F$: for any finite set $ I $ and any map $ f \colon S \to I $, the element $ i = \int_S f \ d \mu \in I $ is the unique one such that $ S_i \in F$.

	The assignments $ \mu \mapsto F_{\mu} $ and $ F \mapsto \mu_F $ together define a bijection between ultrafilters on $S$ and systems of thick subsets.
\end{construction}

\begin{definition}
	If $ S $ is a set, and if $ G \subseteq \PP(S) $, then an ultrafilter $ \mu $ is said to be \defn{supported on $ G $} if and only if every element of $G$ is $ \mu $-thick, that is, $ G \subseteq F_{\mu} $.
\end{definition}

\begin{lemma} \label{generateultrafilters}
	Let $ S $ be a set, and let $ G \subseteq \PP(S) $.
	Assume that no finite intersection of elements of $ G $ is empty.
	Then there exists an ultrafilter $ \mu $ on $ S $ supported on $ G $.
\end{lemma}

\begin{proof}
	Consider all the families $ A \subseteq \PP(S) $ with the following properties:
	\begin{enumerate}
		\item $ A $ contains $ G $;
		\item \label{FIP} no finite intersection of elements of $ A $ is empty.
	\end{enumerate}
	By Zorn's lemma there is a maximal such family, $ F $.

	We claim that $ F $ is a system of thick subsets.
	For this, let $ S = \coprod_{i \in I} S_i $ be a finite partition of $ S $.
	Condition \ref{FIP} ensures that at most one of the summands $ S_i $ can lie in $ F $.
	Now suppose that none of the summands $ S_i $ lies in $ F $.
	Consider, for each $ i \in I $, the family $ F \cup \{ S_i \} \subseteq \PP(S) $;
	the maximality of $ F $ implies that none of these families can satisfy Condition \ref{FIP}.
	Thus for each $ i \in I $, there is an empty finite intersection
	$S_i \cap \bigcap_{j = 1}^{n_i} T_{ij} = \varnothing $.
	But this implies that the intersection $ \bigcap_{i \in I}\bigcap_{j = 1}^{n_i} T_{ij} $ is empty, contradicting Condition \ref{FIP} for $ F $ itself.
	Hence at least one -- and thus exactly one -- of the summands $ S_i $ lies in $ F $.
	Thus $ F $ is a system of thick subsets of $ S $.
\end{proof}

\begin{nul}
	It is not quite accurate to say that the Axiom of Choice is \emph{necessary} to produce nonprincipal ultrafilters, but it is true that their existence is independent of Zermelo--Fraenkel set theory.
\end{nul}

\begin{nul}
	If $ \phi $ is a functor $ \Set \to \Set $, then a natural transformation $ \phi \to \beta $ is the same thing as a natural transformation $ \phi \circ i \to i $.
	Please observe that we have a canonical identification $ \beta \circ i = i $.

	It follows readily that the functor $ \beta $ is a monad: the unit $ \delta \colon \id \to \beta $ corresponds to the identification $ {\id} \circ i = i $, and the multiplication $ \mu \colon \beta^2 \to \beta $ corresponds to the identification $ \beta^2 \circ i = i $.

	The unit for the monad $\beta$ structure is the assignment $ s \mapsto \delta_s $ that picks out the principal ultrafilter at a point.

	To describe the multiplication $ \tau \mapsto \mu_{\tau} $, let us write $ T^{\dag} $ for the set of ultrafilters supported on $\{T\}$.
	Now if $ \tau $ is an ultrafilter on $ \beta(S) $, then $ \mu_{\tau} $ is the ultrafilter on $S$ such that
	\[
		\mu_{\tau} ( T ) = \tau ( T^{\dag} ) \period
	\]
\end{nul}

%-------------------------------------------------------------------%
\subsection{Completeness of ultrafilters}%
\label{sub:completeness_of_ultrafilters}
%-------------------------------------------------------------------%

%-------------------------------------------------------------------%
\subsection{Ultrafilters on posets}%
\label{sub:ultrafilters_on_posets}
%-------------------------------------------------------------------%

%-------------------------------------------------------------------%
\subsection{Ultraproducts}%
\label{sub:ultraproducts}
%-------------------------------------------------------------------%




