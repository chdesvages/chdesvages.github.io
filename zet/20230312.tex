%!TeX program=xelatex
\documentclass[leqno]{article}
\usepackage[normal]{boilerart}
\usepackage[margin=1.5in]{geometry}

\title{Condensed tannakian duality:
Context and questions}
\author{C. Barwick}
\date{12 March 2023}

\begin{document}

\maketitle

My goal is just to ask this question:
\begin{qst*}
    How should condensed tannakian duality work?
\end{qst*}

\section{Pontryagin duality}%
\label{pontryagin}

Let \(A\) be a locally compact abelian (LCA) group.
The group of characters or dual group of \(A\) is then the group
\(A^{\vee} \coloneq \Hom(A,\TT) \comma\)
where \(\TT \subset \CC^{\times}\) is the circle group.
We give this the compact-open topology.
It turns out that \(A^{\vee}\) is also a locally compact abelian group,
and so the functor \(A \to A^{\vee}\) is a functor from
the category of LCA groups to itself.


The basic statement of Pontryagin duality is that this functor
is its own inverse: \(A^{\vee\vee} \simeq A\) naturally in \(A\).
Thus the category \(\categ{LCA}\) is equivalent to its own opposite
via this functor.

Here are some LCA groups and their duals:
\begin{itemize}
    \item finite cyclic groups \(\ZZ/n\) (with the discrete topology) are their own duals;
    \item \(\ZZ\) with the discrete topology is dual to \(\TT\);
    \item any finite-dimensional vector space \(V\) over \(\RR\)
        is an LCA group, and its dual is
        its dual vector space \(V^{\vee}\);
    \item any closed subgroup \(B \subset A\) of an LCA group is LCA,
        and the dual of \(B\) is the quotient \(A^{\vee}/B^{\perp}\)
        by the \emph{annihilator} \(B^{\perp}\) of \(B\) in
        \(A^{\vee}\) -- i.e., the group of characters on \(A\) that
        kill the elements of \(B\);
    \item \(\QQ\) with the discrete topology
        (not the subspace topology!)
        is an LCA group;
        its dual is called the \emph{solenoid},
        which is the limit in topological groups of
        all finite coverings of \(\TT\) by itself;
    \item the \emph{adic circle} \(\QQ/\ZZ\),
        which is the torsion subgroup of \(\TT\) with the discrete topology,
        is an LCA group,
        and careful contemplation of the previous two examples
        will let you deduce that its dual is the group \(\widehat{\ZZ}\) of adic integers;
    \item the Prüfer group \(\QQ_p/\ZZ_p\),
        which is the \(p\)-power-torsion subgroup of \(\TT\) with the discrete topology,
        is an LCA group,
        and its dual group is the group \(\ZZ_p\) of
        \(p\)-adic integers;
    \item it turns out that \(\QQ_p\) is self-dual,
        which is pretty neat;
        that's because you can think of \(\QQ_p\) both as
        a colimit of a limit of \(\ZZ/p^k\)'s and
        a limit of a colimit of \(\ZZ/p^k\)'s.
\end{itemize}

So Pontryagin duality says that an LCA group can be recovered from
a small piece of its category of representations.
The group \(A^{\vee}\) consists of
the unitary, \(1\)-dimensional representations of \(A\).

I learned a rather disturbing proof of this statement
from a short paper of Mike Barr.
Here's my memory of how that goes.

Start with the category \(\Ab\) of abelian groups.
Here I mean \emph{discrete abelian groups} -- no topology.
And no finiteness conditions.
We're going to start by considering \(\TT\) as an object of \(\Ab\).
This is already sounding like a bad idea, I know.
But, ok, we'll write \(|\TT|\) for that object.

Now we're going to perform what is called the \emph{Chu construction}.
This is a category \(\categ{Chu}\) whose objects are
triples \((A,B,\eta)\) in which \(A\) and \(B\) are abelian groups, and
\begin{equation*}
    \eta \colon A \otimes B \to |\TT|
\end{equation*}
is a \emph{nondegenerate pairing}, i.e.,
a homomorphism of abelian groups such that the two homomorphisms
\begin{equation*}
    A \to \Hom(B, |\TT|) \quad \text{and} \quad B \to \Hom(A, |\TT|)
\end{equation*}
are each injective.
A morphism of \(\categ{Chu}\)
\begin{equation*}
    (A, B, \eta) \to (C, D, \theta)
\end{equation*}
is a pair \((\alpha,\delta)\)
consisting of a map \(\alpha \colon C \to A\)
and a map \(\delta \colon B \to D\) such that
for every \(x\in C, y \in B\),
\begin{equation*}
    \eta(\alpha(x),y) = \theta(x,\delta(y)) \period
\end{equation*}

The category \(\categ{Chu}\) has a symmetric monoidal structure,
which is given by the formula
\begin{equation*}
    (A, B, \eta) \otimes (C, D, \theta) \coloneq (\Hom(B,C) \otimes_{\Hom(B \otimes D, |\TT|)} \Hom(D, A), B \otimes D, \xi) \comma
\end{equation*}
where \(\xi\) is the evaluation map.
The unit for this multiplication is \((|\TT|,\ZZ,\id\).

You can see already that this construction is set up so that
\(\categ{Chu}\) is equivalent to its own opposite:
the duality is \((A,B,\eta) \mapsto (B,A,\eta)\).
This defines an internal Hom in \(\categ{Chu}\):
\begin{equation*}
    \Hom((A,B,\eta),(C,D,\theta)) \coloneq (B,A,\eta) \otimes (C,D,\theta) \period
\end{equation*}
Even so, you could be forgiven for wondering what this construction,
which mentions topology exactly nowhere, has to do with Pontryagin duality.

Let's say that a topology on abelian group \(A\) is \defn{admissible}
if it endows \(A\) the structure of a topological group that
can be expressed as a topological subgroup
of a product of LCA groups.

Notice that if \((A,B,\eta)\) is an object of \(\categ{Chu}\), then
we can endow the group \(B\) with the subspace topology from
\begin{equation*}
    \Hom(A,\TT) \simeq \prod_{a \in A} \TT \period
\end{equation*}
This is an admissible topology on \(B\), but
it is not quite the correct topology for our purposes.
So we call this the \defn{naive} topology on \(B\).

Let \(A\) be an abelian group.
Suppose \(\tau_{1}\) and \(\tau_{2}\) are two topologies on \(A\),
each of which makes \(A\) into an admissible topological group.
Then we say that \(\tau_{1}\) and \(\tau_{2}\) are \defn{character-equivalent} if
a homomorphism \(A \to |\TT|\) is continuous for \(\tau_{1}\)
if and only if it is continuous for \(\tau_{2}\).
One can show that for every admissible topology \(\tau\) on \(A\),
there exists a finest topology \(\tau^{'}\)
that is character-equivalent to \(\tau\).
We'll call such a topology (i.e., an admissible topology that
is the finest topology in its character-equivalence class)
\defn{good}.

Now if \((A,B,\eta)\) is an object of \(\categ{Chu}\), then
there is a unique good topology on \(B\) that is character-equivalent
to the naive topology on \(B\).
This defines a functor from \(\categ{Chu}\)
to good topological abelian groups.
In the other direction, if \(A\) is a good topological abelian group,
then \(\Hom(A,\TT),|A|,\epsilon\) is an object of \(\categ{Chu}\),
where \(\epsilon\) is evaluation.
This functor is inverse to the functor above.

In other words, we've identified this abstractly-defined category
with the category of good topological abelian groups.
We can now lift the duality \(\categ{Chu}^{\op} \simeq \categ{Chu}\)
to the category of good topological abelian groups.
One can prove that every LCA group is good, and
from this one can deduce that the duality functor
on good topological abelian groups restricts
to the \enquote{usual} one on LCA groups.

Let's think about how the Pontryagin duality story lifts to spectra.

\section{Brown--Comenetz duality}%
\label{browncomenetz}

\enquote{The} \defn{Brown--Comenetz dual} of the sphere spectrum,
which is usually denoted \(\II\), is characterized up to equivalence
by the formula
\begin{equation*}
    \pi_n \HOM(E, \II) \cong \Hom(\pi_{-n} E, \QQ/\ZZ) \period
\end{equation*}
Here, \(\QQ/\ZZ\) is the adic circle,
which is good enough for Pontryagin duality of finite groups.
This spectrum is unique up to a connected
(but not contractible) space of choices.

The homotopy groups of \(\II\) are \(0\) in positive degrees,
\(\pi_{0}\II = \QQ/\ZZ\), and
\(\pi_{n}\II = (\pi_{-n}\SS)^{\vee}\) for negative \(n\).

If \(E\) is a spectrum whose homotopy groups are all finite,
then you can prove that the natural map
\begin{equation*}
    E \mapsto \HOM(\HOM(E, \II), \II)
\end{equation*}
is an equivalence.
We call \(\HOM(E,\II)\) the \defn{Brown--Comenetz dual} of \(E\).

Notice the funny tension here:
the dual of \(\II\) is not quite \(\SS\), because
\begin{equation*}
    \pi_{0}\HOM(\II,\II) = \Hom(\QQ/\ZZ, \QQ/\ZZ)
    = \widehat{\ZZ} \period
\end{equation*}

Notice also that the Brown--Comenetz dual is different
from the Spanier--Whitehead dual,
which is \(\HOM(E,\SS)\).
Spanier--Whitehead duality works for finite spectra,
which are of course entirely different from spectra
with finite homotopy groups.
(These two types of duality become related in an interesting way
after passage to the \(K(n)\)-local category.)

Already here pyknotic structures seem to provide something new.
Let's call a pyknotic spectrum \(E\) \defn{locally compact}
if its (pyknotic) homotopy groups are all LCA groups.
Note that locally compact spectra include all spectra!

Now let's \enquote{fix} the Brown--Comenetz dual.
\enquote{The} \defn{Pontryagin dual} of the sphere spectrum, which
(in an act of extreme indolence)
I'll denote \(\II\TT\),
is the pyknotic spectrum
characterized up to equivalence by the formula
\begin{equation*}
    \pi_n \HOM(E, \II\TT) \cong \Hom(\pi_{-n} E, \TT) \period
\end{equation*}
Here, \(\TT\) is the \emph{actual} circle group.
Once again this spectrum is unique up to a connected
space of choices.

Now, with no work whatsoever,
we have shown that \(E \mapsto \HOM(E, \II\TT)\) is a
duality functor on these locally constant spectra.
On spectra whose homotopy groups are finite,
this duality extends the Brown--Comenetz duality.
On other spectra (like \(\SS\) itself), it fixes it.

\begin{qst}
    Does the Chu construction from the previous section
    make sense for spectra?
    If so, does it contain these locally compact spectra?
\end{qst}

\section{Cartier duality}%
\label{cartier}

There is a natural refinement of Pontryagin duality to the realm
of \emph{group schemes}, which is called \defn{Cartier duality}.
Here, the role of \(\TT\) is replaced with
the multiplicative group \(\GG_m\).
Over a base ring \(k\), recall, we have
\(\GG_{m,k}(R) = R^{\times}\) for any \(k\)-algebra \(R\).

A \defn{character}
of a commutative group scheme \(A\)
over \(k\)
is a homomorphism of group schemes \(A \to \GG_{m,k}\).
Just as the set of characters of an LCA group
can be endowed with a group topology,
the set of characters of a commutative group scheme
can be endowed with a commutative group scheme structure.
Here's how: for any \(k\)-algebra \(R\),
we set
\begin{equation*}
    A^{\vee}(R) \coloneq
    \Hom_R(A \times_{\Spec k} \Spec R, \GG_{m,R}) \period
\end{equation*}
This is the \defn{Cartier dual} of \(A\).

If \(A\) is a finite flat commutative affine group scheme over \(k\),
then so is \(A^{\vee}\), and moreover, the obvious morphism
\(A \to A^{\vee\vee}\) is an isomorphism.
Thus the category of finite flat commutative affine group schemes
is self-dual.

(Jacob proves a variant of Cartier duality in his
\emph{Elliptic Cohomology} paper.
If I were being more thorough,
I'd include something about that here.) 

So far, we've been focused on abelian structures.
However, it makes sense to ask whether nonabelian groups
can be reconstructed from some of their representations.
That's where Tannaka comes in.

\section{Tannaka--Krein duality}%
\label{tannakakrein}

By definition, a representation of a topological group \(G\) will
always refer to a \emph{continuous} representation.
A \defn{unitary} representation of \(G\) will mean a representation
of \(G\) on a Hilbert space (over \(\CC\))
through unitary transformations.

The Peter--Weyl theorem in representation theory implies that
if \(G\) is \emph{compact}, then
every unitary representation of \(G\) is a direct sum
of irreducible representations,
and every irreducible unitary representation of \(G\) is
finite-dimensional.

So, let's consider the category \(\Rep(G)\) of
finite-dimensional complex representations
of our compact group \(G\).
We can ask:
how much information about \(G\) is stored in this category?
Tannaka gave us an answer:
if you remember the tensor product of representations, all of it.
Here's how that story goes.

We have \(\Rep(G)\), and the tensor product of representations
(which makes sense, since our representations are finite-dimensional)
endows this category with a symmetric monoidal structure.
Furthermore, the forgetful functor from \(\Rep(G)\) to \(\Vect\)
is symmetric monoidal.
For \emph{reasons}, this functor is usually denoted \(\omega\).

Tannaka's idea (I'm perhaps updating his language slightly here)
was to remember what happens if you take a discrete group \(\Gamma\)
and the category \(\Gamma\Set\) of, erm, \(\Gamma\)-sets.
There, you can recover \(\Gamma\) as the automorphisms
of the forgetful functor \(\Gamma\Set \to \Set\).
In effect, if \(\gamma \in \Gamma\), then
you get an automorphism of the forgetful functor
by acting on each \(\Gamma\)-set by \(\gamma\).

So here, we want to consider the group
\begin{equation*}
    \Aut^{\otimes}(\omega)
\end{equation*}
of automorphisms of \(\omega\) as a symmetric monoidal functor.
Every element \(g \in G\) gives us one of these.
(In the case of \(\Gamma\Set \to \Set\),
the only sensible symmetric monoidal structure is \(\times\),
which is automatically preserved by any automorphism.)

This is almost the right thing.
The only additional observation we need is that
if \(g \in G\), then the automorphism \(g \colon \omega \to \omega\)
we obtain is self-conjugate:
that is, if \(\sigma\) denotes complex conjugation, then
\(\sigma \circ g = g\).
So we'll only look at those, and it turns out that
\begin{equation*}
    G \cong \Aut^{\otimes}(\omega)^{\sigma=\id} \comma
\end{equation*}
as groups.

But, ok. What about the topology on \(G\)?
Can that be recovered from the categorical picture as well?
Indeed it can!
We can endow \(\Aut^{\otimes}(\omega)\) with
the coarsest topology such that the maps
\begin{equation*}
    \Aut^{\otimes }(\omega) \to \GL(V)
\end{equation*}
for \(V \in \Rep(G)\) are all continuous.
Transporting this topology to \(G\) given the isomorphism above,
we recover \(G\) as a topological group.

Now notice that
if you have \emph{any} additive \(\CC\)-linear category \(A\)
endowed with a symmetric monoidal structure \(\otimes\),
a semilinear involution \(\sigma\),
and a symmetric monoidal
\(\CC\)-linear functor \(\omega \colon A \to \Vect(\CC)\),
then you can extract a topological group
\(\Aut^{\otimes}(\omega)^{\sigma=\id}\).
What Krein showed next was this:
if \(A\) is semisimple, then this topological group is compact,
and \(\omega\) factors through an equivalence
\begin{equation*}
    A \simeq \Rep(\Aut^{\otimes}(\omega)^{\sigma=\id}) \period
\end{equation*}

Together, these two things are known as
\defn{Tannaka--Krein duality},
which appears to violate Boyer's law of eponymy.

Something remarkable has happened here.
This form of duality finds an equivalence of data between
two different categorical levels.
These sorts of transcategorical dualities are invariably interesting.

One thing to observe here is the role of \(\omega\),
which is called the \defn{fiber functor}.
That functor in some sense is trying to show you \enquote{where}
the identity for the product on \(G\) is.
You could look at \emph{other} choices of fiber functor.
These choices form
a \emph{torsor} for \(G\).

Let's quickly recall this notion.
Let \(C\) be a category with all finite products;
let \(G\) be a group object in \(C\).
Then a \(G\)-\defn{torsor} is a \(G\)-object in \(C\)
such that the \enquote{shear map}
\begin{equation*}
    G \times X \to X \times X \comma \qquad
    (g,x) \mapsto (gx, x)
\end{equation*}
is an isomorphism.
For example, a big chunk of your linear algebra class
could have been summarized by saying that
if \(V\) is an \(n\)-dimensional vector space over a field \(k\),
then the set of bases of \(V\) is a \(\GL_n(k)\)-torsor.
Of course you can't tell undergraduates that,
because you'll scare them.

Grothendieck, being Grothendieck, found a way
to transport all these ideas to the realm of algebraic geometry.

\section{Tannakian duality}%
\label{tannakian}

So let's start with a field \(k\).
We are interested in \(k\)-linear, abelian categories.
We'll equip them with a symmetric monoidal structure
that is additive separately in each variable.
We'll call these \defn{tensor categories} over \(k\).

We'll be particularly interested in \defn{rigid} tensor categories
over \(k\).
These are the tensor categories that are closed symmetric monoidal;
this ensures that the dual \(X^{\vee} \coloneq \HOM(X,\mathbf{1})\)
exists and gives a description of the internal Hom:
\(\HOM(X,Y) \cong X^{\vee}\otimes Y\).

If \(A\) is a rigid tensor category over \(k\), then
an important role will be played by the (commutative) \(k\)-algebra
\(\End(\mathbf{1})\).
For example, let \(X\) be an object, and
let \(f \in \End(X)\) be an endomorphism.
Then we may regard \(f\) as a morphism
\(\mathbf{1} \to X^{\vee} \otimes X\),
and by composing this with the evaluation morphism
\(X^{\vee} \otimes X \to  \mathbf{1}\),
we obtain an element \(\mathop{tr}(f) \in \End(\mathbf{1})\),
which is called the \defn{trace} of \(f\).
In particular, the trace of the identity is called
the \defn{rank} of \(X\).

Let \(A\) be a \(k\)-linear rigid tensor category
such that \(\End(\mathbf{1}) = k\), along with
an exact, symmetric monoidal \defn{fibre functor}
\begin{equation*}
    \omega \colon A \to \Vect(k) \period
\end{equation*}
Now let's imitate the construction of Tannaka--Krein.
We want to take tensor automorphisms of this functor.

But what kind of group do we expect to get from this?
Before we were trying to topologize our group.
In this context, our goal is actually
to get a \emph{scheme structure} on this group.
The functorial attitude to algebraic geometry is that
most \enquote{naturally defined} schemes represent nice functors.
So when we want to define a scheme, we really want
to write down a functor \(\categ{Ring} \to \Set\),
and prove that it is of the form \(R \mapsto \Mor(\Spec R, X)\)
for some scheme \(X\).
And to an extent, if you've written down the \enquote{right} functor,
then the geometry is assured to be interesting.
Since what we actually want here is a \emph{group scheme over \(k\)},
what we're going to do is write down a functor
\(\categ{Ring}_k \to \categ{Grp}\).
Morally, what we want to do for each \(k\)-algebra \(R\) is
to identify automorphisms of our fiber functor \enquote{over \(R\)}.

Ok, so how do we do that?
For each \(k\)-algebra \(R\),
we can tensor up the \(\Hom\) vector spaces of \(A\).
That gives us an \(R\)-linear category \(A \otimes_k R\).
It now comes with a lovely \(R\)-linear functor
\begin{equation*}
    \omega \otimes_k R \colon A \otimes_k R \to \Proj(R) \period
\end{equation*}
We'll just take automorphisms of that!

So we define
\begin{equation*}
    \mathbf{Aut}^{\otimes}(\omega) \colon
    \categ{Ring}_k \to \categ{Grp} \comma \qquad
    R \mapsto \Aut_R^{\otimes}(\omega \otimes_k R) \period
\end{equation*}
This graceful construction turns out to be an affine group scheme
over the field \(k\).

In the other direction, if you have an affine group scheme \(G/k\),
then you can form its category of representations;
this is
\begin{equation*}
    \categ{Rep}(G) \coloneq \Proj(BG) \comma
\end{equation*}
where \(BG\) is the classifying stack.
That is, \(BG\) is the functor \(\categ{Ring} \to \Gpd\)
given by the assignment
\(R \mapsto B\Aut_R^{\otimes}(\omega \otimes_k R)\).
For any \(X \colon \categ{Ring} \to \Cat\), we define
\begin{equation*}
    \Proj(X) \coloneq \Fun_{\Fun(\categ{Ring},\Cat)}(X, \Proj) \comma
\end{equation*}
where you think of \(\Proj\) as the functor \(R \mapsto \Proj(R)\)
(covariant in \(R\)).
Unpacking \(BG\) as the realization of the bar construction,
we obtain
\begin{equation*}
    \Proj(BG) = \lim_{n \in \Delta} \Proj(G^n) \period
\end{equation*}
(Since we're talking about \(1\)-categories at the moment,
this formula is overkill,
because you can restrict \(n\) to \(\Delta_{\leq 2}\),
but this is showing us what kinds of constructions we're going
to have to give in the not-too-distant future.)

Note that the role of \(G\) in this definition
isn't quite as important as that of the stack \(BG\).
One way to think of \(BG\) is as the \defn{stack of fiber functors},
\(\Fib(\omega) \colon \categ{Ring} \to \Gpd\),
which assigns to any \(R\) the groupoid of
\(R\)-linear, symmetric monoidal, exact functors
\(A \otimes_k R \to \Proj_R\)
and symmetric monoidal \(R\)-linear isomorphisms between them.
(In fact, it turns out that all symmetric monoidal
\(R\)-linear natural transformations between such functors are
isomorphisms.)
The condition \(\End(\mathbf{1}) = k\) ensures that
the groupoid \(\Fib(\omega)(k)\) is connected;
thus the specification of a fiber functor \(\omega\) gives rise to
an equivalence 
\begin{equation*}
    \Fib(\omega) \cong B\Aut^{\otimes}(\omega) \period
\end{equation*}

Ok, let's see what we have:
if \(A\) us a \(k\)-linear rigid tensor category,
then for every object \(E \in A\),
you can construct a natural transformation \(\Fib(A) \to \Proj\)
by the assignment
\(\omega \otimes_k R \mapsto (\omega \otimes_k R)(E)\).
That defines a functor \(A \to \Proj(\Fib(A))\).

In the other direction, we can define a morphism
\(X \to \Fib(\Proj(X))\) as follows.
For every \(k\)-algebra \(R\) and every point \(x \in X(R)\),
we obtain a fiber functor \(\xupperstar \colon \Proj(X) \to \Proj(R)\).
This defines an adjunction
\begin{equation*}
    \adjto{\Fib}{\categ{Tens}_k}
    {\Fun(\categ{Ring}_k,\Gpd)^{\op}}{\Proj} \period
\end{equation*}

The big theorem of tannakian duality
(Grothendieck, Saavedra--Rivano, and then Deligne)
is that this adjunction restricts to an equivalence of categories
between the full subcategories consisting of
those rigid tensor categories over \(k\) with \(\End(\mathbf{1})=k\) and the stacks of the form \(BG\).

\section{Higher tannakian duality}%
\label{highertannakian}

I don't know for sure who saw the potential
in higher tannakian duality first.
It seems to have arisen in a conversation between
Carlos Simpson and Bertrand Toën toward the end of the 1990s.
For me, these were the ideas that finally convinced me that
derived algebraic geometry would be important.

The idea is that the tannakian duality theorem above
is tied to the presence of suitable tensor (abelian) categories.
But what if, instead, you only have access to
a symmetric monoidal stable category that is supposed to function
like the derived category?
What sort of object could the group scheme
of automorphisms of the fiber functor actually be?
What sort of object could the stack of fiber functors be?

To answer these questions, let's try to strip away
as much of the exoskeleton of this story as we can.
At a very primitive level, we have the assignment
\(R \mapsto \Mod(R)\), which goes from \enquote{commutative rings}
(of the appropriate kind)
to \enquote{symmetric monoidal categories}
(of the appropriate kind).
In the presence of enough limits
in our category of symmetric monoidal categories,
the functor \(\Mod\) right-Kan-extends to an adjunction
\begin{equation*}
    \adjto{\categ{Fib}}{\Cat^{\otimes}}
    {\Fun(\categ{Ring},\Space)^{\op}}{\QCoh} \comma
\end{equation*}
where \(\categ{Fib}\) is the Yoneda embedding
\(\Cat^{\otimes} \to \Fun(\Cat^{\otimes},\Space)\)
precomposed with the functor \(\Mod\).
We are interested in restricting this adjunction
in such a way as to obtain an equivalence of categories.

In a way, it seems more natural to push this idea a bit further,
\((\infty,2)\)-right-Kan-extending to an adjunction
\begin{equation*}
    \adjto{\categ{FIB}}{\Cat^{\otimes}}
    {\Fun(\categ{Ring},\Cat)^{\op}}{\QCoh} \comma
\end{equation*}
where the left adjoint is the \((\infty,2)\)-categorical
Yoneda embedding, precomposed with the functor \(\Mod\).
One reason this setting is less-explored is that
\enquote{categorical prestacks} \(\categ{Ring} \to \Cat\)
aren't thought to have
much in the way of geometric content.
(Perhaps contemplating \enquote{stratified stacks} offers
a way to think about such gadgets in geometric terms.)

\begin{qst}
    Is there a reasonably large category \(\Cat^{\otimes}\) for which
    the adjunction \((\categ{FIB},\QCoh)\) is \defn{idempotent}?
    Idempotentce means that we have, for every \(\AA\), 
    \begin{equation*}
        \categ{FIB}(\AA) \simeq
        \categ{FIB}(\QCoh(\categ{FIB}(\AA))) \period
    \end{equation*}
    Equivalently, this means we have, for every prestack \(X\),
    \begin{equation*}
        \QCoh(X) \simeq 
        \QCoh(\categ{FIB}(\QCoh(X))) \period
    \end{equation*}
    If this does happen, then there is an equivalence of categories
    between certain categorical prestacks and
    certain symmetric monoidal categories.
    Which ones?
\end{qst}

What compatibility between these two adjunctions obtains?
The \((\infty,2)\)-categorical version of \(\QCoh\) simply extends
the right adjoint \(\QCoh\), but the left adjoints
\(\categ{Fib}\) and \(\categ{FIB}\) are liable to be different.
How might they coincide?

They'll coincide on \(\AA \in \Cat^{\otimes}\) if and only if
the category
\begin{equation*}
    \Cat^{\otimes}(\AA, \Mod(R))
\end{equation*}
is a groupoid for every ring \(R\).
Well, let's consider two objects of this category,
\(\omega_{1}, \omega_{2} \colon \AA \to \Mod(R)\).
How might we ensure that every symmetric monoidal
natural transformation
\(\alpha \colon \omega_{1} \to \omega_{2}\)
is an equivalence?

Notice that if \(A \in \AA\) is strongly dualizable,
then so are \(\omega_{1}(A)\) and \(\omega_{2}(A)\), and
\begin{equation*}
    \omega_{1}(A^{\vee}) \simeq \omega_{1}(A)^{\vee}
    \quad \text{and} \quad
    \omega_{2}(A^{\vee}) \simeq \omega_{2}(A)^{\vee} \period
\end{equation*}
So now we can consider the symmetric monoidal
natural transformation
\begin{equation*}
    \alpha^{\vee} \colon \omega_{2}^{\vee} \to \omega_{1}^{\vee}
    \comma
\end{equation*}
which, when restricted to the full subcategory
\(\AA^{\perf} \subseteq \AA\)
spanned by the strongly dualizable objects,
and composed with the equivalence
\(\AA^{\perf} \simeq \AA^{\perf,\op}\),
becomes a natural transformation
\begin{equation*}
    \omega_{2}|_{\AA^{\perf}}
    \to
    \omega_{1}|_{\AA^{\perf}} \comma
\end{equation*}
which is an inverse to \(\alpha|_{\AA^{\perf}}\).

What that means is that on \(\AA^{\perf}\),
the natural transformation \(\alpha\) is already an equivalence.
So if, for example, the symmetric monoidal functors
that we contemplate in \(\Cat^{\otimes}\) are
\enquote{controlled} by their restrictions
to the strongly dualizable objects, then the category
\(\Cat^{\otimes}(\AA,\Mod(R))\) is indeed a groupoid.

For example, if \(k\) is an \(E_{\infty}\) ring,
then we could consider a category whose objects are
stable, \(k\)-linear, symmetric monoidal,
dualizably generated categories \(\AA\).
That last condition means that
the unit in \(\AA\) is required to be compact, and
\(\AA^{\perf}\) generates \(\AA\) under filtered colimits.
The functors we consider between such categories will be
the symmetric monoidal left adjoints.
As we have now seen,
\(2\)-morphisms in this category are all invertible.

On the other side of the adjunction,
we can contemplate the prestacks
\(X \colon \categ{Ring}_{k/} \to \Space\)
with the property that \(\QCoh(X) \in \categ{Tens}_k\).
I'm tempted to call these \defn{preperfect} prestacks.
At this level of structure,
I still don't know any general tannakian theorems.
(This might be ignorance on my part.)

The tannakian theorems I know are restricted
to prestacks on \emph{connective} \(E_{\infty}\)-rings.
If \(R\) is connective, then \(\Mod(R)\) has a natural t-structure.
If \(R \to S\) is a morphism of connective \(E_{\infty}\)-rings,
then the functor \(\Mod(R) \to \Mod(S)\) is right t-exact.
As a result, this structure often extends
to a t-structure on \(\QCoh(X)\).

For example, let \(X\) be a preperfect prestack
(on \emph{connective} \(E_{\infty}\)-rings only)
that satisfies descent with respect to the fpqc topology and
admits a flat covering by representables.
Such an \(X\) is called a \defn{perfect stack}.
In this case, the \enquote{value-wise connective} objects
of \(\QCoh(X)\) define a t-structure.
These connective objects are clearly preserved by the functors
induced by morphisms of stacks, so we obtain an adjunction
\begin{equation*}
    \adjto{\categ{Fib}}{\categ{Tens}^{t}_k}
    {\categ{PerfStack}^{\op}}{\QCoh} \period
\end{equation*}
\defn{Tannakian duality for perfect stacks},
which is due to Bhatt and Halpern--Leinster,
is the assertion that the right adjoint is fully faithful.
(This is all in SAG Chapter 9.)

\begin{qst}
    How does one characterize those tensor categories of the form
    \(\QCoh(X)\) for a perfect stack \(X\)?
    In some sense, the statement has to be that there are
    \enquote{enough} fiber functors.
    In the abelian case, the existence of fiber functors
    in characteristic zero follows from the existence
    of certain exterior powers of elements,
    by a theorem of Deligne;
    in characteristic \(p\), they follow from the existence
    of certain \enquote{Frobenius powers},
    by a quite recent theorem of Coulembier.
\end{qst}

By the way, not every stack you want in your life is perfect.
For example, if we're over an algebraically closed field
of characteristic \(p\), then \(B\GG_a\) is not perfect.
In fact, \(\QCoh(B\GG_a)\) doesn't even have any compact objects.
There are tannakian duality theorems that
account for these examples as well.
These involve a different description of
the class of symmetric monoidal functors we take.

\begin{qst}
    Is there a tannakian duality theorem for \emph{nonconnective}
    stacks, and without the presence of t-structures?
\end{qst}

\section{Condensed tannakian duality}%
\label{condensedtannakian}

Ok, so now we ask the main question.
Can one combine the \enquote{topological} and \enquote{geometric}
(higher) forms of tannakian duality?

In place of the category \(\categ{Ring}\),
we need to select a suitable category of condensed rings,
and in place of \(\Cat^{\otimes}\),
we need to select a suitable category of
condensed symmetric monoidal categories

A sensible option: take the category of
analytic rings (or \(E_{\infty}\)-rings).
Consider the functor \(R \mapsto \Mod(R)\) that
carries such a ring to its condensed category
of solid modules.
(Recall that an analytic ring is a condensed ring
\emph{equipped} with an attached category of solid modules.)

Why would such a variant be a good idea?
On one hand, we should be able to restrict to discrete objects
to recover the various forms of \enquote{usual} tannakian duality.
On the other hand, any topological group \(G\)
gives rise to a condensed group
(which we also call \(G\)),
and this in turn gives rise to a
\enquote{constant analytic group scheme} \(G_R\)
over the analytic ring \(R\).

\begin{qst}
    How does the theory of analytic group schemes go?
    Is there a variant of Cartier duality at this level?
    In particular, the connection between Pontryagin duality
    and Cartier duality is that the circle \(\TT\)
    sits inside \(\CC^{\times} = \GG_m(\CC)\)
    as a topological subgroup.
    In this analytic setting, how should
    Pontryagin duality for \(G\) compare with
    Cartier duality for \(G_R\)?
\end{qst}

Note that the analytic story is really adapted
to the setting of nonarchimedean fields.
In other words, this analytic tannakian duality theorem,
whatever it is,
is probably not going to be enough to generalize Tannaka--Krein.

\begin{qst}
    Which of the Tannakian duality stories above generalize
    to this setting?
\end{qst}

\end{document}
